\documentclass{article}

% ============ MACRO DEFINITIONS ============
\newcommand{\important}[1]{\textbf{\textcolor{red}{#1}}}
\newcommand{\highlight}[1]{\textit{\textcolor{blue}{#1}}}
\newcommand{\note}[2]{\textbf{#1:} \textit{#2}}

\begin{document}

% ============ SECTION 1 ============
\section{Introduction to Quantum Computing}
\label{sec:intro}

Quantum computing represents a \important{paradigm shift} in computational theory. Unlike classical computers that use bits, quantum computers leverage:
\begin{itemize}
    \item Superposition of quantum states
    \item Entanglement between qubits
    \item Interference patterns for computation
    \item Probabilistic measurement outcomes
\end{itemize}

The fundamental principles discussed in \ref{sec:principles} enable algorithms that outperform classical approaches.

% ============ SECTION 2 ============
\section{Core Principles}
\label{sec:principles}

\subsection{Quantum State Representation}
\label{subsec:states}

The \highlight{quantum state} of a qubit exists in a superposition:
\begin{enumerate}
    \item Linear combinations of basis states
    \item Complex probability amplitudes
    \item Normalization constraints
    \item Measurement collapse
\end{enumerate}

Section \ref{sec:principles}, subsection \ref{subsec:states} establishes the foundation for understanding quantum gates.

\subsection{Key Properties}

Three essential quantum phenomena:
\begin{itemize}
    \item \texttt{Superposition} - qubits exist in \important{multiple states simultaneously}
    \item \texttt{Entanglement} - produces \highlight{correlated measurement outcomes}  
    \item \texttt{Interference} - enables \note{Quantum Speedup}{constructive and destructive interference patterns}
\end{itemize}

\section{Mathematical Framework}
\label{sec:math}

The Schrödinger equation governs quantum evolution:
$$i\hbar\frac{\partial}{\partial t}|\psi\rangle = \hat{H}|\psi\rangle$$

Measurement probability follows the Born rule:
$$P(x) = |\langle x|\psi\rangle|^2$$

These equations form the mathematical backbone of quantum mechanics.

\section{Quantum Gate Comparison}
\label{sec:gates}

\begin{center}
\begin{tabular}{|l|c|r|}
\hline
Gate & Qubits & Reversible \\
\hline
Hadamard & 1 & Yes \\
CNOT & 2 & Yes \\
Toffoli & 3 & Yes \\
\hline
\end{tabular}
\end{center}

\section{Implications and Future Directions}

Quantum computing builds on concepts from \ref{sec:intro}, \ref{sec:principles}, \ref{sec:math}, and \ref{sec:gates}.

\important{Current research focuses on:}
\begin{itemize}
    \item \highlight{Error correction} schemes to maintain coherence
    \item \important{Scalable architectures} for practical quantum computers
    \item Quantum supremacy demonstrations
    \item Hybrid classical-quantum algorithms
\end{itemize}

\end{document}
